% !TEX root = template.tex

\section{Introduction}
\label{sec:introduction}

Activity recognition systems are promising for the next-generation technologies, they will be used both for entertainment scenarios and to improve some aspects of the medical and survival sector.
Existing solutions are usually implemented extracting hand-crafted features used as input for classifiers such as \gls{svm} \cite{Elvira14, Hamalainen11, Khan10}.
This approach is consolidated and it leads to good results in terms of accuracy of the predictions, although hand-crafted features are data dependent and could not be generalized for different application domains.
In the last few years a lot of effort has been put in implementing good \gls{ars} using non feature-dependent techniques to have a more general model and to reuse it in different scenarios.

This work improves the work made in \cite{Frank10} using a \gls{cnn}-based technique to predict the proposed activities.
In the original work a total of 19 features were extracted from the recorded signals, making the model strongly data-dependent.
The aim of this work is to elaborate the dataset used in \cite{Frank10} and to learn a more general model to predict human activities in real time with a good accuracy presenting more complete results.

Given their 2D nature, \glspl{cnn} are usually applied to imaging field, such as the prediction of diseases classifying x-rays images or the recognition of object, people and animals.
This work applies \glspl{cnn} to 1D signals, proving that these kind of Neural Networks are not limited to 2D signals but they can have a wide range of applications.
Moreover, since this approach does not require an ad-hoc dataset or a particular sensor to work, it can be applied (with some little changes) to any dataset with the same purpose.

Summing up what has been done in this work, the main contributions are reported in the following:
\begin{itemize}
\item a more general model \gls{cnn}-based is used, to make this work reusable allowing other researchers to improve the architecture here implemented
\item \gls{cnn} is used in a non-typical 1D scenario, this proves the flexibility of this tool and the adaptability to a very wide range of applications
\item more complete results regarding the accuracy reached in the test phase are presented, showing the behaviour of the model in situations where few data are available and these have an high variance.
\end{itemize}

The paper is structured as follows. In section \ref{sec:related_work} the state-of-art literature is presented, in section \ref{sec:processing_architecture} are reported, at large, the main steps made by the implemented \gls{ars} to predict the activities.
In section \ref{sec:model} the signals collected in the dataset are described and the pre-processing algorithm to make them suitable for the learning framework is explained in detail.
The learning framework is presented in \ref{sec:cnn_architecture} and the final results are commented in \ref{sec:results}.
Finally in \ref{sec:conclusions} are reported the difficulty faced during the development of the system and conclusions are drawn.