% !TEX root = template.tex

\section{Processing Pipeline}
\label{sec:processing_architecture}

\MR{I would start the technical description with a {\it high level} introduction of your processing pipeline. Here you do not have to necessarily go into the technical details of every processing block, this will be done later as the paper develops. What I would like to see here is a description of the general approach, i.e., which processing blocks you used, how these were concatenated, etc. A diagram usually helps.}

\section{Signals and Features}
\label{sec:model}

\MR{Being a machine learning paper, I would put here a section describing the signals you have been working on. If possible, you should describe, in order, 1) the measurement setup, 2) how the signals were \mbox{pre-processed} (to remove noise, artifacts, fill gaps or represent them through a constant sampling rate, etc.). After this, you should describe how {\it feature vectors} were obtained from the \mbox{pre-processed} signals. If signals are {\it time series} this also implies stating the segmentation / windowing strategy that was adopted, to then describe how you obtained a feature vector for each time window. Also, if you also experiment with previous feature extraction approaches, you may want to list them as well, in addition to (and before) your own (possibly new) proposal.}

\section{Learning Framework}
\label{sec:learning_framework}

\MR{Here you finally describe the learning strategy / algorithm that you conceived and used to solve the problem at stake. A good diagram to exemplify how learning is carried out is often very useful. In this section, you should describe the learning model, its parameters, any optimization over a given parameter set, etc. You can organize this section in \mbox{sub-sections}. You are free to choose the most appropriate structure.} 